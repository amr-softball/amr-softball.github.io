\subsection*{Introduction}

This document explains how to use the mushy layer code for various physical applications.

It assumes you already have a compiled version of the code which you can run.

\subsection*{Basics}

There are over a hundred options used in the code, which are stored on the {\ttfamily \hyperlink{class_a_m_r_level_mushy_layer}{A\+M\+R\+Level\+Mushy\+Layer}} class in one of two places\+:


\begin{DoxyEnumerate}
\item The member variable {\ttfamily \hyperlink{class_a_m_r_level_mushy_layer_abe361484bd04d3be84f4c3a0c96d896d}{A\+M\+R\+Level\+Mushy\+Layer.\+m\+\_\+parameters}} which is a {\ttfamily \hyperlink{class_mushy_layer_params}{Mushy\+Layer\+Params}} object. This is where most physical parameters (e.\+g. Rayleigh number, Lewis number) are stored.
\item The member variable {\ttfamily \hyperlink{class_a_m_r_level_mushy_layer_a3652e6785ac8c5e429a5ac286ea3bc2e}{A\+M\+R\+Level\+Mushy\+Layer.\+m\+\_\+opt}}, which is a struct of type {\ttfamily \hyperlink{struct_mushy_layer_options}{Mushy\+Layer\+Options}} and contains options on how to solve the equations (e.\+g. solver thresholds, how to do projection, and lots of debugging options).
\end{DoxyEnumerate}

Inputs are loaded in the code via the {\ttfamily \textbf{ Parm\+Parse}} class, e.\+g.


\begin{DoxyCode}
Real multigrid\_threshold = 1e-10;
ParmParse pp(\textcolor{stringliteral}{"main"});
pp.query(\textcolor{stringliteral}{"mg\_thresh"}, multigrid\_threshold)
\end{DoxyCode}
 which would load some inputs like 
\begin{DoxyCode}
main.mg\_thresh=1e-15
\end{DoxyCode}


We do most of this in either {\ttfamily /mushy-\/layer/src\+Subcycle/\+Mushy\+Layer\+Subcycle\+Utils.cpp} or {\ttfamily /mushy-\/layer/src/\+Mushy\+Layer\+Params.cpp}.

Most of these options/parameters can just be left at their default values. The inputs file at {\ttfamily /mushy-\/layer/examples/meltponds/inputs\+Grow\+Sea\+Ice} illustrates the main options you may wish to change, which are described in more detail below under some roughly suitable headings.

\subsection*{General options}

{\ttfamily main.\+verbosity=2} larger verbosity means there will be more text output produced

{\ttfamily main.\+output\+\_\+folder=.} folder to save plot/checkpoint files to a period {\ttfamily .} means the directory from which the code was executed

{\ttfamily main.\+plot\+\_\+prefix=plt} prefix for plot files, which will then be like {\ttfamily plt001234.\+2d.\+hdf5}

{\ttfamily main.\+chk\+\_\+prefix=chk} prefix for checkpoint files

{\ttfamily main.\+plot\+\_\+interval=n} produce plot files every {\ttfamily n} steps (-\/1 = don\textquotesingle{}t use this option)

{\ttfamily main.\+checkpoint\+\_\+interval=n} produce checkpoints files every {\ttfamily n} steps

{\ttfamily main.\+plot\+\_\+period=0.\+005} time interval at which to write out plot files

{\ttfamily main.\+debug=false} set to true to write more fields to the plot files

\subsection*{Timestepping}

{\ttfamily main.\+cfl=0.\+1} max allowed C\+FL number

{\ttfamily main.\+initial\+\_\+cfl=2e-\/05} start simulations with a smaller C\+FL number. The timestep increases thereafter as determined by {\ttfamily main.\+max\+\_\+dt\+\_\+growth}

{\ttfamily main.\+max\+\_\+dt=0.\+1} max allowed timestep

{\ttfamily main.\+max\+\_\+dt\+\_\+growth=1.\+05} max fractional increase in \$ t\$ allowed from one timestep to the next

{\ttfamily main.\+max\+\_\+step=10000} stop simulations after this many timesteps

{\ttfamily main.\+max\+\_\+time=10.\+0} stop simulations once this time has been reached

{\ttfamily main.\+min\+\_\+time=0.\+1} ensure simulations run until at least this time (even if a steady state condition is reached)

{\ttfamily main.\+dt\+\_\+tolerance\+\_\+factor=1.\+01} Set the factor by which the current dt must exceed the new (max) dt for time subcycling to occur (i.\+e., reduction of the current dt by powers of 2).

\subsection*{Domain setup}

{\ttfamily main.\+num\+\_\+cells=Nx Ny Nz} e.\+g. {\ttfamily main.\+num\+\_\+cells=64 128} size of the grid on the coarsest mesh. Make sure each dimensions is a power of 2 unless you want the multigrid to be slow. Only need to specify 2 dimensions if the code is compiled for 2 dimensions. The final dimension is always the \textquotesingle{}vertical\textquotesingle{} with respect to gravity.

{\ttfamily main.\+domain\+\_\+height=1.\+0} domain height (width is then calculated from the number of grid cells specified)

{\ttfamily main.\+periodic\+\_\+bc=1 0} whether to enforce periodic boundaries (1) or not (0) in each dimension

{\ttfamily main.\+max\+\_\+grid\+\_\+size=256} split domain into boxes (for sending to different processors) of max size in any dimension given by this parameter.

\subsection*{Initial and boundary conditions}

See the separate document.

\subsection*{Dimensionless parameters}

{\ttfamily main.\+nondimensionalisation=0} choose nondimensionalisation, different options are\+: 0. Diffusive timescale, advective velocity scale
\begin{DoxyEnumerate}
\item Darcy timescales, advective velocity scale
\item Darcy timescale, darcy velocity scale
\item Advective timescale, darcy velocity scale
\item Buoyancy timescale, advective velocity scale
\end{DoxyEnumerate}

{\ttfamily parameters.\+problem\+\_\+type=0} can specify different problem types, which can tell the code to solve some of the equations differently. Lots of the options aren\textquotesingle{}t really used anymore (and maybe don\textquotesingle{}t work). Ones which definitely should work are in bold. 0. {\itshape mushy\+Layer}
\begin{DoxyEnumerate}
\item burgers equation
\item burgers equation on a periodic domain
\item poiseuille flow
\item diffusion
\item {\itshape solidification\+No\+Flow}
\item corner flow
\item sidewall heating
\item vertical heating (with Darcy\textquotesingle{}s equation)
\item rayleigh-\/benard (navier-\/stokes)
\item a solute flux test
\item a reflux test
\item a test for solving with zero porosity
\item simulating a melting ice block
\item {\itshape convection\+Mixed\+Porous}
\item vortex pair (navier-\/stokes benchmark problem)
\end{DoxyEnumerate}

{\ttfamily parameters.\+composition\+Ratio=1.\+18} composition ratio, should be $>$= 1.\+0.

{\ttfamily parameters.\+rayleigh\+Comp=500} rayleigh number for compositional differences. When just solving Darcy\textquotesingle{}s equation, this is the mushy layer number. Otherwise, it\textquotesingle{}s the fluid rayleigh number.

{\ttfamily parameters.\+rayleigh\+Temp=25.\+0} rayleigh number for temperature.

{\ttfamily parameters.\+stefan=5.\+0} stefan number

{\ttfamily parameters.\+K=1} ratio of solid/liquid heat diffusivity

{\ttfamily parameters.\+heat\+Conductivity\+Ratio=1} ratio of solid/liquid heat conductivity

{\ttfamily parameters.\+specific\+Heat\+Ratio=1} ratio of solid.\+liquid specific heat

{\ttfamily parameters.\+lewis=100} lewis number (liquid heat diffusivity/liquid salt diffusivity)

{\ttfamily parameters.\+darcy=0.\+0} darcy number. Set to 0 to solve Darcy\textquotesingle{}s equation rather than Darcy-\/\+Brinkman

{\ttfamily parameters.\+prandtl=0} prandtl number

{\ttfamily parameters.\+hele\+Shaw=true} constrain permeabilities by a Hele-\/\+Shaw cell

{\ttfamily parameters.\+non\+Dim\+Reluctance=0.\+1} inverse of the dimensionless Hele-\/\+Shaw permeabilty. Smaller means wider hele-\/shaw gap width. Alternatively, you can now specify the dimensionless Hele-\/\+Shaw cell permeability e.\+g. {\ttfamily parameters.\+hele\+Shaw\+Permeability=10.\+0}

{\ttfamily parameters.\+non\+Dim\+Vel=0.\+0} dimensionless frame advection velocity {\ttfamily parameters.\+permeability\+Function=2} mathematical form of the permeability function. Options include\+: 0. pure\+Fluid (permeability=1)
\begin{DoxyEnumerate}
\item cubic (permeability=porosity$^\wedge$3)
\item kozeny-\/carman (permeability = porosity$^\wedge$3 / (1-\/porosity)$^\wedge$2
\item log function (permeability = -\/ porosity$^\wedge$2 $\ast$ log(1-\/porosity) )
\item spatially dependent permeability with a channel
\item permeability = porosity
\end{DoxyEnumerate}

{\ttfamily parameters.\+water\+Distribution\+Coeff=1e-\/5} water distribution coefficient

\subsection*{Phase diagram}

{\ttfamily parameters.\+eutectic\+Composition=230}

{\ttfamily parameters.\+eutectic\+Temp=-\/23}

{\ttfamily parameters.\+initial\+Composition=30}

{\ttfamily parameters.\+liquidus\+Slope=-\/0.\+1}

\subsection*{\textbf{ A\+MR} options}

{\ttfamily main.\+max\+\_\+level=0} max allowed level for \textbf{ A\+MR} simulations.

{\ttfamily main.\+block\+\_\+factor=8} all boxes must be divisible by at least this length in all dimensions.

{\ttfamily main.\+use\+\_\+subcycling=1} whether or not to use subcycling in time

{\ttfamily main.\+ref\+\_\+ratio=2 2 2} refinement ratios between successive levels

{\ttfamily main.\+refine\+\_\+thresh=0.\+1} refinement threshold

{\ttfamily main.\+regrid\+\_\+interval=8 8 8} regrid interval on each level

{\ttfamily main.\+tag\+\_\+buffer\+\_\+size=4} number of cells by which to grow the set of cells tagged for refinement, before creating the new grids

{\ttfamily main.\+fill\+\_\+ratio=0.\+75} should be between 0 and 1. A value of 1 means the grids generated on refined levels will be as tightly grouped as possible to the cells tagged for refinement. A smaller value means that Chombo will cover a larger area on refined levels with grid cells.

{\ttfamily main.\+grid\+\_\+buffer\+\_\+size=0 0 0} this is the \textquotesingle{}padding\textquotesingle{} between grids on different levels of refinement

{\ttfamily projection.\+eta=0.\+0} Freestream correction coefficient. Should be less than 1 for stability, but close to 1 for accuracy.

\subsection*{Projection}

The projection solver has some tolerance specified by {\ttfamily projection.\+solver\+Tol}. If the unprojected velocity has some divergence \$d\$, then the projected velocity will have some divergence of order \$d  \${\ttfamily projection.\+solver\+Tol}. In reality, we care more about the absolute value of the final divergence than how much it is has been reduced. Therefore we introduce an option to adapt the solver tolerance to achieve a particular final divergence\+:

{\ttfamily projection.\+adapt\+Solver\+Params\+DivU = 1} turns on this option (set 0, or don\textquotesingle{}t set at all, to not use this option) {\ttfamily projection.\+divergence\+\_\+tolerance = 1e-\/9} absolute divergence we wish to aim for

If you are struggling to achieve this final divergence, you can try increasing either the number of multigrid smooths or number of multigrid iterations\+:

{\ttfamily projection.\+num\+Smooth\+Up=64} {\ttfamily projection.\+max\+Iter=40}

Additionally, you can try using the pressure from the previous timestep to remove a significant ammount of the divergence before projection\+:

{\ttfamily projection.\+use\+Incremental\+Pressure=true}

i.\+e. after computing the unprojected velocity \$\{U\}$^\wedge$$\ast$\$, then subtract off the old pressure gradient to find \$\{U\}$^\wedge$$\ast$ -\/   p\$ before projecting this to find a (hopefully) small extra pressure correction. This can significantly speed up the solve. 