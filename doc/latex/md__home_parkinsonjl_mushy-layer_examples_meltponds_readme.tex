The inputs files in this directory are designed to


\begin{DoxyEnumerate}
\item simulate sea ice growth
\item add a melt pond to some sea ice
\item simulate the melt pond evolution
\end{DoxyEnumerate}

\section*{Prerequisites}

To run the code you\textquotesingle{}ll need to have compiled the code {\ttfamily /mushy-\/layer/exec\+Subcycle/}, and also the code in {\ttfamily /mushy-\/layer/setup\+New\+Run/}


\begin{DoxyCode}
$ cd ../../execSubcycle/; make all
$ cd ../../setupNewRun/; make all
\end{DoxyCode}


\section*{Simulate sea ice growth}

Assuming your compiled mushy layer executable is called {\ttfamily mushy\+Layer2d.\+Linux.\+64.\+mpi\+C\+C.\+gfortran.\+O\+P\+T.\+M\+P\+I.\+ex} (it may not be), you can start growing sea ice via


\begin{DoxyCode}
$ cd ~/mushy-layer/examples/meltponds/
$ ../../execSubcycle/mushyLayer2d.Linux.64.mpiCC.gfortran.OPT.MPI.ex  inputsGrowSeaIce
\end{DoxyCode}


This is a very coarse simulation, which should grow a sensible ammount of ice in a few minutes. Once you\textquotesingle{}re happy that you have enough ice, cancel the execution and move on to the next step.

\section*{Adding a melt pond}

Adding a melt pond is done via a seperate program. You\textquotesingle{}ll need to edit the {\ttfamily setup\+New\+Run\+Add\+Pond} file to ensure that the {\ttfamily in\+File}, {\ttfamily run\+\_\+inputs}, and {\ttfamily outfile} paths are correct. Then run the program via


\begin{DoxyCode}
../../setupNewRun/setupnewrun2d.Linux.64.mpiCC.gfortran.OPT.MPI.ex  setupNewRunAddPond
\end{DoxyCode}


This should create a {\ttfamily restart.\+2d.\+hdf5} file at the location specified in the {\ttfamily setup\+New\+Run\+Add\+Pond} inputs file.

\section*{Simulating melt pond evolution}

Make sure the parameter {\ttfamily main.\+restart\+\_\+file} in {\ttfamily inputs\+Evolve\+Melt\+Pond} points to the newly created file, then start simulations like before but with the new inputs file 
\begin{DoxyCode}
../../execSubcycle/mushyLayer2d.Linux.64.mpiCC.gfortran.OPT.MPI.ex  inputsEvolveMeltPond
\end{DoxyCode}


The key difference between growing ice and evolving the melt pond simulations are\+: 
\begin{DoxyCode}
bc.enthalpyHiVal=0 6.02
parameters.pressureHead=5
bc.velHi=6 9 0 # pressure head top bc
main.plot\_period=0.00002
\end{DoxyCode}
 We increase the enthalpy boundary at the top of the domain in the vertical direction, specify the pressure head, change velocity boundary condition at the top of the domain to use this pressure head, and make the plotting period more frequent (as we now have much stronger flow, so smaller timesteps). 