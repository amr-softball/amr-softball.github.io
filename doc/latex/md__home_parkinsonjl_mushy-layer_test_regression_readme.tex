This directory contains regression tests. The idea is to run a simulation and check that the output hasn\textquotesingle{}t changed (hasn\textquotesingle{}t \textquotesingle{}regressed\textquotesingle{}). Assuming the original output was correct, the program still works.

We compare either the text output (e.\+g {\ttfamily pout.\+0} or {\ttfamily diagnostics.\+csv}) and/or the data output ({\ttfamily .hdf5} files). Each subdirectory contains a test, which at a bare minimum consist of the following files\+:


\begin{DoxyEnumerate}
\item {\ttfamily inputs}\+: the input parameters for the simulation
\item {\ttfamily regression.\+json}\+: the options for running the test (number of dimensions, processors etc.). See an existing file for a template.
\item At least one \textquotesingle{}expected\textquotesingle{} file e.\+g. {\ttfamily pout.\+0.\+expected} or {\ttfamily plt000010.\+2d.\+hdf5.\+expected} which will be compared with the computed output files (less the {\ttfamily .expected} suffix) to check for any deviations.
\end{DoxyEnumerate}

\section*{Prerequisites}


\begin{DoxyEnumerate}
\item Compile the mushy layer program. Needs to be done separately in 2D and 3D for the relevant tests to work.
\item Compile the Chombo Compare utility (at {\ttfamily \$\+C\+H\+O\+M\+B\+O\+\_\+\+H\+O\+ME/lib/utils/\+Chombo\+Compare}) for comparing hdf5 files, and ensure the {\ttfamily C\+H\+O\+M\+B\+O\+\_\+\+H\+O\+ME} environmental variable is set.
\end{DoxyEnumerate}

\section*{Usage}

Run all the tests using the python file {\ttfamily run\+\_\+regression\+\_\+tests.\+py} e.\+g.


\begin{DoxyCode}
~/mushy-layer/test/regression$ python run\_regression\_tests.py
\end{DoxyCode}


Check the output to the terminal, which will report which tests fail and which succeed, along with further information on test failures.

\section*{Modify a test}

If you wish to update the expected results of a test (e.\+g. you have changed the code and the test now \textquotesingle{}fails\textquotesingle{} when it shouldn\textquotesingle{}t) then just replace the existing {\ttfamily $\ast$.expected} files with new ones you have generated.

\section*{Create a new test}

To add a new test, make a new subdirectory in this present directory with the an {\ttfamily inputs} file, {\ttfamily regression.\+json} file (see an existing example for a template) and the {\ttfamily $\ast$.expected} files you wish to test against. The test will be run automatically by the python script (it searches this directory for all subdirectories).

Please try and ensure that you keep the test runtime to a manageable length ($\sim$10 seconds). The idea is that it is possible to run all the tests in a few minutes, to encourage users to do so regularly following changes to the code (and especially before committing code). 